\begin{frame}
  \frametitle{rootfs in memory: initramfs (1)}
  \begin{itemize}
  \item It is also possible to have the root filesystem integrated
    into the kernel image
  \item It is therefore loaded into memory together with the kernel
  \item This mechanism is called {\bf initramfs}
    \begin{itemize}
    \item It integrates a compressed archive of the filesystem into
      the kernel image
    \item Variant: the compressed archive can also be loaded separately
      by the bootloader.
    \end{itemize}
  \item It is useful for two cases
    \begin{itemize}
    \item Fast booting of very small root filesystems. As the
      filesystem is completely loaded at boot time, application
      startup is very fast.
    \item As an intermediate step before switching to a real root
      filesystem, located on devices for which drivers not part of the
      kernel image are needed (storage drivers, filesystem drivers,
      network drivers). This is always used on the kernel of
      desktop/server distributions to keep the kernel image size
      reasonable.
    \end{itemize}
  \end{itemize}
\end{frame}

\begin{frame}
  \frametitle{rootfs in memory: initramfs (2)}
  \begin{center}
    \includegraphics[width=0.9\textwidth]{slides/initramfs/initramfs.pdf}
  \end{center}
\end{frame}

\begin{frame}
  \frametitle{rootfs in memory: initramfs (3)}
  \begin{itemize}
  \item The contents of an initramfs are defined at the kernel
    configuration level, with the \code{CONFIG_INITRAMFS_SOURCE}
    option
    \begin{itemize}
    \item Can be the path to a directory containing the root
      filesystem contents
    \item Can be the path to a cpio archive
    \item Can be a text file describing the contents of the initramfs \\
      (see documentation for details)
    \end{itemize}
  \item The kernel build process will automatically take the contents
    of the \code{CONFIG_INITRAMFS_SOURCE} option and integrate the
    root filesystem into the kernel image
  \item Details (in kernel sources): \\
    {\small
    \kerneldoc{filesystems/ramfs-rootfs-initramfs.txt}\\
    \kerneldoc{early-userspace/README}
    }
  \end{itemize}
\end{frame}
