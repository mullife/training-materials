\begin{frame}
  \frametitle{From Bootloader to user space}
  \begin{center}
    \includegraphics[width=\textwidth]{slides/kernel-init-content/bootloader-userspace.pdf}
  \end{center}
\end{frame}

\begin{frame}[fragile]
  \frametitle{Kernel Bootstrap (1)}
  How the kernel bootstraps itself appears in kernel building.
  Example on ARM (pxa cpu) in Linux 2.6.36:
{\footnotesize
\begin{verbatim}
...
  LD      vmlinux
  SYSMAP  System.map
  SYSMAP  .tmp_System.map
  OBJCOPY arch/arm/boot/Image
  Kernel: arch/arm/boot/Image is ready
  AS      arch/arm/boot/compressed/head.o
  GZIP    arch/arm/boot/compressed/piggy.gzip
  AS      arch/arm/boot/compressed/piggy.gzip.o
  CC      arch/arm/boot/compressed/misc.o
  CC      arch/arm/boot/compressed/decompress.o
  AS      arch/arm/boot/compressed/head-xscale.o
  SHIPPED arch/arm/boot/compressed/lib1funcs.S
  AS      arch/arm/boot/compressed/lib1funcs.o
  LD      arch/arm/boot/compressed/vmlinux
  OBJCOPY arch/arm/boot/zImage
  Kernel: arch/arm/boot/zImage is ready
...
\end{verbatim}
}
\end{frame}

\begin{frame}
  \frametitle{Kernel Bootstrap (2)}
  \begin{center}
    \includegraphics[width=\textwidth]{slides/kernel-init-content/kernel-bootstrap.pdf}
  \end{center}
\end{frame}

\begin{frame}
  \frametitle{Bootstrap Code for Compressed Kernels}
  \begin{itemize}
  \item Located in \code{arch/<arch>/boot/compressed}
    \begin{itemize}
    \item \code{head.o}
      \begin{itemize}
      \item Architecture specific initialization code.
      \item This is what is executed by the bootloader
      \end{itemize}
    \item \code{head-cpu.o} (here \code{head-xscale.o})
      \begin{itemize}
      \item CPU specific initialization code
      \end{itemize}
    \item \code{decompress.o}, \code{misc.o}
      \begin{itemize}
      \item Decompression code
      \end{itemize}
    \item \code{piggy.<compressionformat>.o}
      \begin{itemize}
      \item The kernel itself
      \end{itemize}
    \end{itemize}
  \item Responsible for uncompressing the kernel itself and jumping to
    its entry point.
  \end{itemize}
\end{frame}

\begin{frame}
  \frametitle{Architecture-specific Initialization Code}
  \begin{itemize}
  \item The uncompression code jumps into the main kernel entry point,
    typically located in \code{arch/<arch>/kernel/head.S}, whose job
    is to:
    \begin{itemize}
    \item Check the architecture, processor and machine type.
    \item Configure the MMU, create page table entries and enable
      virtual memory.
    \item Calls the \code{start_kernel} function in
      \code{init/main.c}.
    \item Same code for all architectures.
    \item Anybody interested in kernel startup should study this file!
    \end{itemize}
  \end{itemize}
\end{frame}

\begin{frame}
  \frametitle{start\_kernel Main Actions}
  \begin{itemize}
  \item Calls \code{setup_arch(&command_line)}
    \begin{itemize}
    \item Function defined in \code{arch/<arch>/kernel/setup.c}
    \item Copying the command line from where the bootloader left it.
    \item On arm, this function calls \code{setup_processor} (in which
      CPU information is displayed) and \code{setup_machine}(locating
      the machine in the list of supported machines).
    \end{itemize}
  \item Initializes the console as early as possible (to get error
    messages)
  \item Initializes many subsystems (see the code)
  \item Eventually calls \code{rest_init}.
  \end{itemize}
\end{frame}

\begin{frame}[fragile]
  \frametitle{rest\_init: Starting the Init Process}
\begin{minted}[fontsize=\tiny]{c}
static noinline void __init_refok rest_init(void)
        __releases(kernel_lock)
{
        int pid;

        rcu_scheduler_starting();
        /*
         * We need to spawn init first so that it obtains pid 1, however
         * the init task will end up wanting to create kthreads, which, if
         * we schedule it before we create kthreadd, will OOPS.
         */
        kernel_thread(kernel_init, NULL, CLONE_FS | CLONE_SIGHAND);
        numa_default_policy();
        pid = kernel_thread(kthreadd, NULL, CLONE_FS | CLONE_FILES);
        rcu_read_lock();
        kthreadd_task = find_task_by_pid_ns(pid, &init_pid_ns);
        rcu_read_unlock();
        complete(&kthreadd_done);

        /*
         * The boot idle thread must execute schedule()
         * at least once to get things moving:
         */
        init_idle_bootup_task(current);
        preempt_enable_no_resched();
        schedule();
        preempt_disable();

        /* Call into cpu_idle with preempt disabled */
        cpu_idle();
}
\end{minted}
\end{frame}

\begin{frame}[fragile]
  \frametitle{kernel\_init}
  \begin{itemize}
  \item \code{kernel_init} does two main things:
    \begin{itemize}
    \item Call \code{do_basic_setup}
    \item Once kernel services are ready, start device initialization
      (Linux 2.6.36 code excerpt):
\begin{minted}{c}
static void __init do_basic_setup(void)
{
    cpuset_init_smp();
    usermodehelper_init();
    init_tmpfs();
    driver_init();
    init_irq_proc();
    do_ctors();
    do_initcalls();
}
\end{minted}
    \item Call \code{init_post}
    \end{itemize}
  \end{itemize}
\end{frame}

\begin{frame}[fragile]
  \frametitle{do\_initcalls}
Calls pluggable hooks registered with the macros below.
Advantage: the generic code doesn't have to know about them.
\begin{minted}[fontsize=\tiny]{c}
/*
 * A "pure" initcall has no dependencies on anything else, and purely
 * initializes variables that couldn't be statically initialized.
 *
 * This only exists for built-in code, not for modules.
 */
#define pure_initcall(fn)               __define_initcall("0",fn,1)

#define core_initcall(fn)               __define_initcall("1",fn,1)
#define core_initcall_sync(fn)          __define_initcall("1s",fn,1s)
#define postcore_initcall(fn)           __define_initcall("2",fn,2)
#define postcore_initcall_sync(fn)      __define_initcall("2s",fn,2s)
#define arch_initcall(fn)               __define_initcall("3",fn,3)
#define arch_initcall_sync(fn)          __define_initcall("3s",fn,3s)
#define subsys_initcall(fn)             __define_initcall("4",fn,4)
#define subsys_initcall_sync(fn)        __define_initcall("4s",fn,4s)
#define fs_initcall(fn)                 __define_initcall("5",fn,5)
#define fs_initcall_sync(fn)            __define_initcall("5s",fn,5s)
#define rootfs_initcall(fn)             __define_initcall("rootfs",fn,rootfs)
#define device_initcall(fn)             __define_initcall("6",fn,6)
#define device_initcall_sync(fn)        __define_initcall("6s",fn,6s)
#define late_initcall(fn)               __define_initcall("7",fn,7)
#define late_initcall_sync(fn)          __define_initcall("7s",fn,7s)
\end{minted}
Defined in \code{include/linux/init.h}
\end{frame}

\begin{frame}[fragile]
  \frametitle{initcall example}
From \kfileversion{arch/arm/mach-pxa/lpd270.c}{2.6.36} (Linux 2.6.36)
\begin{minted}[fontsize=\scriptsize]{c}
static int __init lpd270_irq_device_init(void)
{
    int ret = -ENODEV;
    if (machine_is_logicpd_pxa270()) {
        ret = sysdev_class_register(&lpd270_irq_sysclass);
        if (ret == 0)
            ret = sysdev_register(&lpd270_irq_device);
    }
    return ret;
}

device_initcall(lpd270_irq_device_init);
\end{minted}
\end{frame}

\begin{frame}
  \frametitle{init\_post}
  \begin{itemize}
  \item The last step of Linux booting
    \begin{itemize}
    \item First tries to open a console
    \item Then tries to run the init process, effectively turning the
      current kernel thread into the user space init process.
    \end{itemize}
  \end{itemize}
\end{frame}

\begin{frame}[fragile]
  \frametitle{init\_post Code: init/main.c}
\begin{minted}[fontsize=\tiny]{c}
static noinline int init_post(void) __releases(kernel_lock) {
    /* need to finish all async __init code before freeing the memory */
    async_synchronize_full();
    free_initmem();
    mark_rodata_ro();
    system_state = SYSTEM_RUNNING;
    numa_default_policy();

    current->signal->flags |= SIGNAL_UNKILLABLE;
    if (ramdisk_execute_command) {
        run_init_process(ramdisk_execute_command);
        printk(KERN_WARNING "Failed to execute %s\n", ramdisk_execute_command);
    }

    /* We try each of these until one succeeds.
     * The Bourne shell can be used instead of init if we are
     * trying to recover a really broken machine. */
    if (execute_command) {
        run_init_process(execute_command);
        printk(KERN_WARNING "Failed to execute %s.  Attempting defaults...\n", execute_command);
    }
    run_init_process("/sbin/init");
    run_init_process("/etc/init");
    run_init_process("/bin/init");
    run_init_process("/bin/sh");

    panic("No init found. Try passing init= option to kernel. See Linux Documentation/init.txt");
}
\end{minted}
\end{frame}

\begin{frame}
  \frametitle{Kernel Initialization Graph}
  \begin{center}
    \includegraphics[width=\textwidth]{slides/kernel-init-content/kernel-initialization.pdf}
  \end{center}
\end{frame}

\begin{frame}
  \frametitle{Kernel Initialization - Summary}
  \begin{itemize}
  \item The bootloader executes bootstrap code.
  \item Bootstrap code initializes the processor and board, and
    uncompresses the kernel code to RAM, and calls the kernel's
    \code{start_kernel} function.
  \item Copies the command line from the bootloader.
  \item Identifies the processor and machine.
  \item Initializes the console.
  \item Initializes kernel services (memory allocation, scheduling,
    file cache...)
  \item Creates a new kernel thread (future init process) and
    continues in the idle loop.
  \item Initializes devices and execute initcalls.
  \end{itemize}
\end{frame}
