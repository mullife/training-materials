\section{Developing Kernel Modules}

\begin{frame}[fragile]
  \frametitle{Hello Module 1/2}
\begin{minted}[fontsize=\scriptsize]{c}
/* hello.c */
#include <linux/init.h>
#include <linux/module.h>
#include <linux/kernel.h>

static int __init hello_init(void)
{
  pr_alert("Good morrow to this fair assembly.\n");
  return 0;
}

static void __exit hello_exit(void)
{
  pr_alert("Alas, poor world, what treasure hast thou lost!\n");
}

module_init(hello_init);
module_exit(hello_exit);
MODULE_LICENSE("GPL");
MODULE_DESCRIPTION("Greeting module");
MODULE_AUTHOR("William Shakespeare");
\end{minted}
\end{frame}

\begin{frame}[fragile]
  \frametitle{Hello Module 2/2}
\begin{itemize}
\item \ksym{__init}
  \begin{itemize}
  \item removed after initialization (static kernel or module.)
  \end{itemize}
\item \ksym{__exit}
  \begin{itemize}
  \item discarded when module compiled statically into the kernel,
        or when module unloading support is not enabled.
  \end{itemize}
\item Example available on
  \url{http://git.free-electrons.com/training-materials/plain/code/hello/hello.c}
\end{itemize}
\end{frame}

\begin{frame}
  \frametitle{Hello Module Explanations}
  \begin{itemize}
  \item Headers specific to the Linux kernel: \code{linux/xxx.h}
    \begin{itemize}
    \item No access to the usual C library, we're doing kernel
      programming
    \end{itemize}
  \item An initialization function
    \begin{itemize}
    \item Called when the module is loaded, returns an error code
      (\code{0} on success, negative value on failure)
    \item Declared by the \kfunc{module_init} macro: the name of the
      function doesn't matter, even though \code{<modulename>_init()}
      is a convention.
    \end{itemize}
  \item A cleanup function
    \begin{itemize}
    \item Called when the module is unloaded
    \item Declared by the \kfunc{module_exit} macro.
    \end{itemize}
  \item Metadata information declared using \kfunc{MODULE_LICENSE},
    \kfunc{MODULE_DESCRIPTION} and \kfunc{MODULE_AUTHOR}
  \end{itemize}
\end{frame}

\begin{frame}
  \frametitle{Symbols Exported to Modules 1/2}
  \begin{itemize}
  \item From a kernel module, only a limited number of kernel
    functions can be called
  \item Functions and variables have to be explicitly exported by the
    kernel to be visible to a kernel module
  \item Two macros are used in the kernel to export functions and
    variables:
    \begin{itemize}
    \item \code{EXPORT_SYMBOL(symbolname)}, which exports a function
      or variable to all modules
    \item \code{EXPORT_SYMBOL_GPL(symbolname)}, which exports a
      function or variable only to GPL modules
    \end{itemize}
  \item A normal driver should not need any non-exported function.
  \end{itemize}
\end{frame}

\begin{frame}
  \frametitle{Symbols exported to modules 2/2}
  \begin{center}
    \includegraphics[height=0.85\textheight]{slides/kernel-driver-development-modules/exported-symbols.pdf}
  \end{center}
\end{frame}

\begin{frame}
  \frametitle{Module License}
  \begin{itemize}
  \item Several usages
    \begin{itemize}
    \item Used to restrict the kernel functions that the module can
      use if it isn't a GPL licensed module
      \begin{itemize}
      \item Difference between \kfunc{EXPORT_SYMBOL} and
        \kfunc{EXPORT_SYMBOL_GPL}
      \end{itemize}
    \item Used by kernel developers to identify issues coming from
      proprietary drivers, which they can't do anything about
      (“Tainted” kernel notice in kernel crashes and oopses).
    \item Useful for users to check that their system is 100\% free
      (check \code{/proc/sys/kernel/tainted})
    \end{itemize}
  \item Values
    \begin{itemize}
    \item GPL compatible (see \kfile{include/linux/license.h}:
      \code{GPL}, \code{GPL v2}, \code{GPL and additional rights},
      \code{Dual MIT/GPL}, \code{Dual BSD/GPL}, \code{Dual MPL/GPL})
    \item \code{Proprietary}
    \end{itemize}
  \end{itemize}
\end{frame}

\begin{frame}
  \frametitle{Compiling a Module}
  Two solutions
  \begin{itemize}
  \item \emph{Out of tree}
    \begin{itemize}
    \item When the code is outside of the kernel source tree, in a
      different directory
    \item Advantage: Might be easier to handle than modifications to
      the kernel itself
    \item Drawbacks: Not integrated to the kernel
      configuration/compilation process, needs to be built
      separately, the driver cannot be built statically
    \end{itemize}
  \item Inside the kernel tree
    \begin{itemize}
    \item Well integrated into the kernel configuration/compilation
       process
    \item Driver can be built statically if needed
    \end{itemize}
  \end{itemize}
\end{frame}

\begin{frame}[fragile]
  \frametitle{Compiling an out-of-tree Module 1/2}
  \begin{itemize}
  \item The below \code{Makefile} should be reusable for any single-file
    out-of-tree Linux module
  \item The source file is \code{hello.c}
  \item Just run \code{make} to build the \code{hello.ko} file
  \end{itemize}
{\footnotesize
\begin{block}{}
\begin{minted}{make}
ifneq ($(KERNELRELEASE),)
obj-m := hello.o
else
KDIR := /path/to/kernel/sources

all:
<tab>$(MAKE) -C $(KDIR) M=$$PWD
endif
\end{minted}
\end{block}
}

\begin{itemize}
\item \code{KDIR}: kernel source or headers directory (see next slides)
\end{itemize}
\end{frame}

\begin{frame}
  \frametitle{Compiling an out-of-tree Module 2/2}
  \begin{center}
    \includegraphics[height=0.4\textheight]{slides/kernel-driver-development-modules/out-of-tree.pdf}
  \end{center}
  \begin{itemize}
  \item The module \code{Makefile} is interpreted with \code{KERNELRELEASE}
    undefined, so it calls the kernel \code{Makefile}, passing the module
    directory in the \code{M} variable
  \item The kernel \code{Makefile} knows how to compile a module, and thanks
    to the \code{M} variable, knows where the \code{Makefile} for our module
    is. This module \code{Makefile} is then interpreted with \code{KERNELRELEASE}
    defined, so the kernel sees the \code{obj-m} definition.
  \end{itemize}
\end{frame}

\begin{frame}
  \frametitle{Modules and Kernel Version}
  \begin{itemize}
  \item To be compiled, a kernel module needs access to the kernel
    headers, containing the definitions of functions, types and
    constants.
  \item Two solutions
    \begin{itemize}
    \item Full kernel sources
      (configured + \code{make modules_prepare})
    \item Only kernel headers (\code{linux-headers-*} packages in
      Debian/Ubuntu distributions, or directory created by \code{make
      headers_install})
    \end{itemize}
  \item The sources or headers must be configured
    \begin{itemize}
    \item Many macros or functions depend on the configuration
    \end{itemize}
  \item A kernel module compiled against version X of kernel headers
    will not load in kernel version Y
    \begin{itemize}
    \item \code{modprobe} / \code{insmod} will say \code{Invalid module format}
    \end{itemize}
  \end{itemize}
\end{frame}

\begin{frame}[fragile]
  \frametitle{New Driver in Kernel Sources 1/2}
  \begin{itemize}
  \item To add a new driver to the kernel sources:
    \begin{itemize}
    \item Add your new source file to the appropriate source
      directory. Example: \kfile{drivers/usb/serial/navman.c}
    \item Single file drivers in the common case, even if the file is
      several thousand lines of code big. Only really big drivers are
      split in several files or have their own directory.
    \item Describe the configuration interface for your new driver by
      adding the following lines to the \code{Kconfig} file in this
      directory:
    \end{itemize}
{\footnotesize
\begin{block}{}
\begin{verbatim}
config USB_SERIAL_NAVMAN
        tristate "USB Navman GPS device"
        depends on USB_SERIAL
        help
          To compile this driver as a module, choose M
          here: the module will be called navman.
\end{verbatim}
\end{block}
}
  \end{itemize}
\end{frame}

\begin{frame}
  \frametitle{New Driver in Kernel Sources 2/2}
  \begin{itemize}
  \item Add a line in the \code{Makefile} file based on the
\code{Kconfig} setting:
    \code{obj-$(CONFIG_USB_SERIAL_NAVMAN) += navman.o}
  \item It tells the kernel build system to build \code{navman.c} when the
    \code{USB_SERIAL_NAVMAN} option is enabled. It works both if
    compiled statically or as a module.
    \begin{itemize}
    \item Run \code{make xconfig} and see your new options!
    \item Run \code{make} and your new files are compiled!
    \item See \kerneldoc{kbuild/} for details and more
      elaborate examples like drivers with several source files, or
      drivers in their own subdirectory, etc.
    \end{itemize}
  \end{itemize}
\end{frame}

\begin{frame}[fragile]
  \frametitle{Hello Module with Parameters 1/2}
\begin{minted}[fontsize=\small]{c}
/* hello_param.c */
#include <linux/init.h>
#include <linux/module.h>

MODULE_LICENSE("GPL");

/* A couple of parameters that can be passed in: how many
   times we say hello, and to whom */

static char *whom = "world";
module_param(whom, charp, 0);

static int howmany = 1;
module_param(howmany, int, 0);

\end{minted}
\end{frame}

\begin{frame}[fragile]
  \frametitle{Hello Module with Parameters 2/2}
\begin{minted}[fontsize=\footnotesize]{c}
static int __init hello_init(void)
{
  int i;
  for (i = 0; i < howmany; i++)
    pr_alert("(%d) Hello, %s\n", i, whom);
  return 0;
}

static void __exit hello_exit(void)
{
  pr_alert("Goodbye, cruel %s\n", whom);
}

module_init(hello_init);
module_exit(hello_exit);
\end{minted}
Thanks to Jonathan Corbet for the example!\\
Source code available on: {\small
\url{http://git.free-electrons.com/training-materials/plain/code/hello-param/hello_param.c}}
\end{frame}

\begin{frame}[fragile]
  \frametitle{Declaring a module parameter}

\begin{minted}[fontsize=\small]{c}
module_param(
    name, /* name of an already defined variable */
    type, /* either byte, short, ushort, int, uint, long, ulong,
             charp, bool or invbool. (checked at run time!) */
    perm  /* for /sys/module/<module_name>/parameters/<param>,
             0: no such module parameter value file */
);

/* Example */
static int irq=5;
module_param(irq, int, S_IRUGO);
\end{minted}
Modules parameter arrays are also possible with
\kfunc{module_param_array}.
\end{frame}
