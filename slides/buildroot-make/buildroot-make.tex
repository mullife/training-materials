\section{GNU Make 101}

\begin{frame}{Introduction}
  \begin{itemize}
  \item Buildroot being implemented in {\bf GNU Make}, it is quite
    important to know the basics of this language
    \begin{itemize}
    \item Basics of {\em make} rules
    \item Defining and referencing variables
    \item Conditions
    \item Defining and using functions
    \item Useful {\em make} functions
    \end{itemize}
  \item This does not aim at replacing a full course on {\em GNU Make}
  \item \url{http://www.gnu.org/software/make/manual/make.html}
  \item \url{http://www.nostarch.com/gnumake}
  \end{itemize}
\end{frame}

\begin{frame}[fragile]{Basics of {\em make} rules}
  \begin{itemize}
  \item At their core, {\em Makefiles} are simply defining {\bf rules}
    to create {\bf targets} from {\bf prerequisites} using {\bf recipe
      commands}
    \begin{block}{}
\begin{verbatim}
TARGET ...: PREREQUISITES ...
         RECIPE
         ...
\end{verbatim}
    \end{block}
  \item {\bf target}: name of a file that is generated. Can also be an
    arbitrary action, like \code{clean}, in which case it's a {\bf
      phony target}
  \item {\bf prerequisites}: list of files or other targets that are
    needed as dependencies of building the current target.
  \item {\bf recipe}: list of shell commands to create the target from
    the prerequisites
  \end{itemize}
\end{frame}

\begin{frame}[fragile]{Rule example}

\begin{block}{Makefile}
\begin{minted}[fontsize=\scriptsize]{make}
clean:
        rm -rf $(TARGET_DIR) $(BINARIES_DIR) $(HOST_DIR) \
                $(BUILD_DIR) $(BASE_DIR)/staging \
                $(LEGAL_INFO_DIR)

distclean: clean
        [...]
        rm -rf $(BR2_CONFIG) $(CONFIG_DIR)/.config.old \
               $(CONFIG_DIR)/.auto.deps
\end{minted}
\end{block}

\begin{itemize}
\item \code{clean} and \code{distclean} are phony targets
\end{itemize}

\end{frame}

\begin{frame}[fragile]{Defining and referencing variables}
  \begin{itemize}
  \item Defining variables is done in different ways:
    \begin{itemize}
    \item \code{FOOBAR = value}, expanded at time of use
    \item \code{FOOBAR := value}, expanded at time of assignment
    \item \code{FOOBAR += value}, prepend to the variable, with a
      separating space, defaults to expanded at the time of use
    \item \code{FOOBAR ?= value}, defined only if not already defined
    \item Multi-line variables are described using \code{define NAME
        ... endef}:
      \begin{block}{}
\begin{verbatim}
define FOOBAR
line 1
line 2
endef
\end{verbatim}
      \end{block}
    \end{itemize}
  \item Make variables are referenced using the \code{$(FOOBAR)}
    syntax.
  \end{itemize}
\end{frame}

\begin{frame}[fragile]{Conditions}
  \begin{itemize}
  \item With \code{ifeq} or \code{ifneq}
    \begin{block}{}
\begin{minted}[fontsize=\tiny]{make}
ifeq ($(BR2_CCACHE),y)
CCACHE := $(HOST_DIR)/usr/bin/ccache
endif

distclean: clean
ifeq ($(DL_DIR),$(TOPDIR)/dl)
        rm -rf $(DL_DIR)
endif
\end{minted}
    \end{block}
  \item With the \code{$(if ...)} make function:
    \begin{block}{}
\begin{minted}[fontsize=\tiny]{make}
HOSTAPD_LIBS += $(if $(BR2_STATIC_LIBS),-lcrypto -lz)
\end{minted}
    \end{block}
  \end{itemize}
\end{frame}

\begin{frame}[fragile]{Defining and using functions}
  \begin{itemize}
  \item Defining a function is exactly like defining a variable:
    \begin{block}{}
\begin{minted}[fontsize=\tiny]{make}
MESSAGE = echo "$(TERM_BOLD)>>> $($(PKG)_NAME) $($(PKG)_VERSION) $(call qstrip,$(1))$(TERM_RESET)"

define legal-license-header # pkg, license-file, {HOST|TARGET}
        printf "$(LEGAL_INFO_SEPARATOR)\n\t$(1):\
                $(2)\n$(LEGAL_INFO_SEPARATOR)\n\n\n" >>$(LEGAL_LICENSES_TXT_$(3))
endef
\end{minted}
    \end{block}
  \item Arguments accessible as \code{$(1)}, \code{$(2)}, etc.
  \item Called using the \code{$(call func,arg1,arg2)} construct
    \begin{block}{}
\begin{minted}[fontsize=\tiny]{make}
$(BUILD_DIR)/%/.stamp_extracted:
        [...]
        @$(call MESSAGE,"Extracting")

define legal-license-nofiles # pkg, {HOST|TARGET}
        $(call legal-license-header,$(1),unknown license file(s),$(2))
endef
\end{minted}
    \end{block}
  \end{itemize}
\end{frame}

\begin{frame}[fragile]{Useful {\em make} functions}
  \begin{itemize}
  \item \code{subst} and \code{patsubst} to replace text
    \begin{block}{}
\begin{minted}[fontsize=\tiny]{make}
ICU_SOURCE = icu4c-$(subst .,_,$(ICU_VERSION))-src.tgz
\end{minted}
    \end{block}
  \item \code{filter} and \code{filter-out} to filter entries
  \item \code{foreach} to implement loops
\begin{block}{}
\begin{minted}[fontsize=\tiny]{make}
$(foreach incdir,$(TI_GFX_HDR_DIRS),
      $(INSTALL) -d $(STAGING_DIR)/usr/include/$(notdir $(incdir)); \
      $(INSTALL) -D -m 0644 $(@D)/include/$(incdir)/*.h \
              $(STAGING_DIR)/usr/include/$(notdir $(incdir))/
)
\end{minted}
\end{block}
  \item \code{dir}, \code{notdir}, \code{addsuffix}, \code{addprefix}
    to manipulate file names
    \begin{block}{}
\begin{minted}[fontsize=\tiny]{make}
UBOOT_SOURCE = $(notdir $(UBOOT_TARBALL))

IMAGEMAGICK_CONFIG_SCRIPTS = \
        $(addsuffix -config,Magick MagickCore MagickWand Wand)
\end{minted}
    \end{block}
  \item And many more, see the {\em GNU Make} manual for details.
  \end{itemize}
\end{frame}

\begin{frame}[fragile]{Writing recipes}

  \begin{itemize}
  \item Recipes are just shell commands
  \item Each line must be indented with one \code{Tab}
  \item Each line of shell command in a given recipe is independent
    from the other: variables are not shared between lines in the
    recipe
  \item Need to use a single line, possibly split using $\backslash$,
    to do complex shell constructs
  \item Shell variables must be referenced using \code{$$name}.
    \begin{block}{package/pppd/pppd.mk}
\begin{minted}[fontsize=\tiny]{make}
define PPPD_INSTALL_RADIUS
        ...
        for m in $(PPPD_RADIUS_CONF); do \
                $(INSTALL) -m 644 -D $(PPPD_DIR)/pppd/plugins/radius/etc/$$m \
                        $(TARGET_DIR)/etc/ppp/radius/$$m; \
        done
        ...
endef
\end{minted}
\end{block}
  \end{itemize}
\end{frame}
