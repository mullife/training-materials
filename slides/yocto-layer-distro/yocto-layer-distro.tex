\section{Distro Layers}

\subsection{Distro Layers}

\begin{frame}{Distro layers}
  \begin{center}
    \includegraphics[width=\textwidth]{slides/yocto-layer-distro/yocto-layer-distro.pdf}
  \end{center}
\end{frame}

\begin{frame}
  \frametitle{Distro layers}
  \begin{itemize}
    \item You can create a new distribution by using a Distro layer.
    \item This allows to change the defaults that are used by
      \code{Poky}.
    \item It is useful to distribute changes that have been made in
      \code{local.conf}
  \end{itemize}
\end{frame}

\begin{frame}
  \frametitle{Best practice}
  \begin{itemize}
    \item A distro layer is used to provides policy configurations for
      a custom distribution.
    \item It is a best practice to separate the distro layer from the
      custom layers you may create and use.
    \item It often contains:
      \begin{itemize}
        \item Configuration files.
        \item Specific classes.
        \item Distribution specific recipes: initialization scripts,
          splash screen packages\dots
      \end{itemize}
  \end{itemize}
\end{frame}

\begin{frame}[fragile]
  \frametitle{Creating a Distro layer}
  \begin{itemize}
    \item The configuration file for the distro layer is
      \code{conf/distro/<distro>.conf}
    \item This file must define the \code{DISTRO} variable.
    \item It is possible to inherit configuration from an existing
      distro layer.
    \item You can also use all the \code{DISTRO_*} variables.
    \item Use \code{DISTRO = "<distro>"} in \code{local.conf} to use
      your distro configuration.
  \end{itemize}
  \begin{block}{}
    \begin{minted}[fontsize=\small]{sh}
require conf/distro/poky.conf

DISTRO = "distro"
DISTRO_NAME = "distro description"
DISTRO_VERSION = "1.0"

MAINTAINER = "..."
    \end{minted}
  \end{block}
\end{frame}

\begin{frame}
  \frametitle{Toolchain selection}
  \begin{itemize}
    \item The toolchain selection is controlled by the \code{TCMODE}
      variable.
    \item It defaults to \code{"default"}.
    \item The \code{conf/distro/include/tcmode-${TCMODE}.inc} file is
      included.
    \begin{itemize}
      \item This configures the toolchain to use by defining preferred
        providers and versions for packages such as \code{gcc},
        \code{binutils}, \code{*libc}\dots
    \end{itemize}
    \item The providers' recipes define how to compile or/and install
      the toolchain.
    \item Toolchains can be built by the build system or external.
  \end{itemize}
\end{frame}

\begin{frame}
  \frametitle{Sample files}
  \begin{itemize}
    \item A distro layer often contains \code{sample files}, used as
      templates to build key configurations files.
    \item Example of \code{sample files}:
      \begin{itemize}
        \item \code{bblayers.conf.sample}
        \item \code{local.conf.sample}
      \end{itemize}
    \item In \code{Poky}, they are in \code{meta-poky/conf/}.
    \item The \code{TEMPLATECONF} variable controls where to find the
      samples.
    \item It is set in \code{${OEROOT}/.templateconf}.
  \end{itemize}
\end{frame}

\begin{frame}
  \frametitle{Distribute the distribution}
  \begin{itemize}
    \item A good way to distribute a distribution (Poky, custom
      layers, BSP, \code{.templateconf}\dots) is to use Google's
      \code{repo}.
    \item \code{Repo} is used in Android to distribute its source
      code, which is split into many \code{git} repositories. It's a
      wrapper to handle several \code{git} repositories at once.
    \item The only requirement is to use \code{git}.
    \item The \code{repo} configuration is stored in \code{manifest}
      file, usually available in its own \code{git} repository.
  \end{itemize}
\end{frame}

\begin{frame}[fragile]
  \frametitle{Manifest example}
  \begin{minted}[fontsize=\small]{xml}
<?xml version="1.0" encoding="UTF-8"?>
<manifest>
  <remote name="yocto-project" fetch="git.yoctoproject.org" />
  <remote name="private" fetch="git.example.net" />

  <default revision="krogoth" remote="private" />

  <project name="poky" remote="yocto-project" />
  <project name="meta-ti" remote="yocto-project" />
  <project name="meta-custom" />
  <project name="meta-custom-bsp" />
  <project path="meta-custom-distro" name="distro">
    <copyfile src="templateconf" dest="poky/.templateconf" />
  </project>
</manifest>
  \end{minted}
\end{frame}

\begin{frame}[fragile]
  \frametitle{Retrieve the project using \code{repo}}
  \begin{minted}{sh}
$ mkdir my-project; cd my-project
$ repo init -u https://git.example.net/manifest.git
$ repo sync -j4
  \end{minted}
  \begin{itemize}
    \item \code{repo init} uses the \code{default.xml} manifest in the
    repository, unless specified otherwise.
    \item You can see the full \code{repo} documentation at
      \url{https://source.android.com/source/using-repo.html}.
  \end{itemize}
\end{frame}
