\subsection{Source code organization}
\begin{frame}
  \frametitle{Source Code organization 1/3}
  \begin{itemize}
  \item Once the source code is downloaded, you will find several
    folders in it(Example with Android 9)
  \end{itemize}
  \begin{description}
  \item[art/] contains the source code of the ART virtual
    machine
  \item[bionic/] is where Android's standard C library is stored
  \item[bootable/] contains only with a recovery image
  \item[build/] holds the core components of the build system
  \item[compatibility/] document related to CDD
  \item[cts/] The Compatibility Test Suite
   \item[dalvik/] contains the source code of the Dalvik virtual
    machine[\textbf{DROPPED}]
  \end{description}
\end{frame}

\begin{frame}
  \frametitle{Source Code Organization 2/3}
  \begin{description}
  \item[development/] holds the development tools, debug applications,
    API samples, etc
  \item[device/] contains the device-specific components
  \item[docs/] contains HTML documentation hosted at
    \url{http://source.android.com}
  \item[external/] is one of the largest folders in the source code, it
    contains all the external projects used in the Android code
  \item[frameworks/] holds the source code of the various parts of the
    framework
  \item[hardware/] contains all the hardware abstraction layers
  \end{description}
\end{frame}

\begin{frame}
  \frametitle{Source Code Organization 3/3}
  \begin{description}
  \item[libcore/] is the Java core library
  \item[libnativehelper/] contains a few JNI helpers for the Android
    base classes
  \item[ndk/] is the place where you will find the Native Development
    Kit, which allows to build native applications for Android
  \item[packages/] contains the standard Android applications
  \item[prebuilt/] holds all the prebuilt binaries, most notably the
    toolchains
  \item[sdk/] is where you will find the Software Development Kit
  \item[system/] contains all the basic pieces of the Android system:
    init, shell, the volume manager, etc.
  \end{description}
  \begin{itemize}
  \item You can get a more precise description at
    \url{http://elinux.org/Master-android}
  \end{itemize}
\end{frame}
