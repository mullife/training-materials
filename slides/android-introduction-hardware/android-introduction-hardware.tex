\subsection{Hardware Requirements for Android}

\begin{frame}
  \frametitle{Android Hardware Requirements}
  \begin{itemize}
  \item Google produces a document updated every new Android version
    called the Compatibility Definition Document (\emph{CDD}).
  \item This document provides all the information you need on the
    expectations Google have about what should be an Android device
  \item It details both the hardware and the global behaviour of the
    system.
  \item While nothing forces you to follow that document if you don't
    care about the Google applications, it usually gives a good idea
    of the current hardware requirements.
  \item \url{http://source.android.com/compatibility/android-cdd.pdf}
  \end{itemize}
\end{frame}

\begin{frame}
  \frametitle{SoC requirements}
  \begin{itemize}
  \item Since Android in itself is quite huge, the hardware required
    is quite powerful.
  \item Unlike Linux, Android officially supports only a few
    architectures
    \begin{itemize}
    \item ARM v7a (basically, all the SoCs based on the Cortex-A CPUs)
    \item x86
    \item MIPS
    \end{itemize}
  \item You also need to have a powerful enough GPU with OpenGL ES/Vulkan
    support. Latest versions of Android require the 3D hardware
    acceleration
  \end{itemize}
\end{frame}
