\subchapter{Application optimization}{Optimize the startup time of your
applications}

\section{Measuring}

The general rule stays the same. You have to measure the time taken to
execute the various pieces of code in your application.

Here, except for possible compiler optimization, modifying the
application is outside the scope of this workshop, because it requires
a good knowledge about the application itself.

However, we are going to use a few techniques which should help you to
improve your own application when you are back to real life.

\subsection{Compiling utilities}

With a build system like Buildroot, it's easy to add performance
analysis and debugging utilities.

Configure Buildroot to add \code{strace} to your root
filesystem. You will find the corresponding configuration option in
\code{Package selection for the target} and then in \code{Debugging,
profiling and benchmark}.

Note that with this version of Buildroot and Linux 3.6.9, we didn't
manage to compile the \code{perf} utility. We will try again when a
newer stable kernel is available for this board.

Run Buildroot and reflash your device as usual.

\subsection{Tracing and profiling with strace}

With \code{strace}'s help, you can already have a pretty good understanding
of how your application spends it time. You can see all the system
calls that it makes and knowing the application, you can guess in which
part of the code it is at a given time.

You can also spot unnecessary attempts to open files that do not exist,
multiple accesses to the same file, or more generally things that the
program was not supposed to do. All these correspond to opportunities
to fix and optimize your application.

Once the board has booted, run \code{strace} on the video player
application:

\begin{verbatim}
strace -tt -f -o strace.log /root/go.sh
\end{verbatim}

Also have \code{strace} generate a summary:

\begin{verbatim}
strace -c -f -o strace-summary.log /root/go.sh
\end{verbatim}

Take some time to read \code{strace.log}\footnote{
At this stage, when you have to open files directly on the
board, some familiarity with the basic commands of the \code{vi} editor
becomes useful. See
\url{http://free-electrons.com/doc/command_memento.pdf} for a basic
command summary. Otherwise, you can use the more rudimentary \code{more}
command. You can also copy the files to your PC, using a USB drive, for
example.}
, and look for the time when the program opens the video file.

Also have a look at \code{strace-summary.log}. You will find the number
of errors trying to open files that do not exist, for example. You can
also count the number of memory allocations (using the \code{mmap2} system call).

\section{Removing unnecessary functionality}

Based on what you learned by tracing and profiling your application, you
could recompile it to remove support for the features that you know are
not used in your system. This should speed up its execution, at least
slightly.

In our case, we could recompile GStreamer with just the features
and plugins required to play the exact video format and codec that we
have.

\section{Postponing, reordering}

In our particular case, modifying \code{gst-launch} would be very
difficult. It could make sense with your own application though, for
which the code is familiar to you.

\section{Optimizing necessary functionality}

In this particular case, \code{gst-launch} is far from being the most
efficient way of opening the video. It is really meant to help creating
a multimedia pipeline. Once it is well defined, the \code{GStreamer}
developers recommend to directly program with the \code{GStreamer}
API\footnote{See the details on
\url{http://docs.gstreamer.com/display/GstSDK/Basic+tutorial+10\%3A+GStreamer+tools}}.
